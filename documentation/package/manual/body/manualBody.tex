\documentclass[8pt]{extbook}
\usepackage{epsfig,fullpage,times,graphs}

\renewcommand{\rmdefault}{phv}

\usepackage[dvips]{geometry}

\newlength{\paperWidth}
\setlength{\paperWidth}{4.125in}
\newlength{\paperHeight}
\setlength{\paperHeight}{4.125in}

% trim a bit more to reduce bottom whitespace
%\setlength{\paperHeight}{0.987\paperHeight}

\geometry{papersize={\paperWidth,\paperHeight},scale=1.0, top=0.25in, bottom=0.375in, left=0.25in, right=0.25in, headheight=0in, headsep=0in, footskip=0.1875in}
%total={6.5in,5in}}


% force all sections to start on new pages
%\let\stdsection\section  
%\renewcommand\section{\newpage \stdsection} 


\setcounter{tocdepth}{0}

\begin{document}

\frontmatter

\tableofcontents


\chapter{Introduction}

Diamond Trust of London has been a very long time coming, and it's too much of an uneven story to fully relate here.  Suffice it to say that what you hold in your hands right now exists only because of several unlikely and fortunate events.  

A turn-based deception game about diamond traders in Angola made entirely by two lifelong friends and released in a box for the Nintendo DS.  I can't believe it either, but it exists.  

I hope you enjoy playing it as much as I enjoyed making it.

\begin{flushright}
\begin{minipage}{1.5in}
Jason Rohrer\\
Las Cruces, New Mexico\\
August 2011
\end{minipage}
\end{flushright}

\mainmatter

\part{How to play}
\addtocounter{chapter}{1}
\setcounter{section}{0}

\section{Overview}
Diamond Trust is extremely simple on a basic mechanical level.  However, it's a new game design that does not share too many play patterns with other games that you've played before.  When you sit down at new platformer or a new realtime strategy game or a new first person shooter, you already know how to play.  That won't be the case with Diamond Trust---you must learn how to play.

First things first:  your goal is to {\bf have more diamonds than your opponent by the end of the game}.  How you use the mechanics to accomplish that goal is up to you.

Each game is divided into nine rounds, and during each round, play proceeds through a series of phases.  In each phase, you and your opponent make your move selection simultaneously and in secret.  Then both moves are revealed, the outcomes of the moves are applied, and the game advances to the next phase.

The mechanics at your disposal in the various phases include moving your agents from region to region, bidding on diamonds, moving the UN Inspector to change which regions are open for play, and selling some of your collected diamonds to raise additional money for future rounds.

On top of these basic mechanics, there is a layer of agent loyalty mechanics.  You can pay your agents to keep them loyal and bribe your opponent's agents to acquire their loyalty as well.  When an agent is loyal to the other side, suddenly the ``make your move in secret'' guarantee goes out the window.  Of course, you won't know that your agent has been compromised, unless you manage to bribe the agent that bribed your agent.  And all of this paying and bribing reduces the amount of money you have to bid on diamonds.  You really can't afford to do it.  Then again, you can't afford {\it not} to do it either.

\section{Structure of the game}

The diagram on page \pageref{fig:gameFlow} shows the phase progression during each round of the game.  That progression is repeated nine times until the end of the game.  Sections \ref{sec:payPhase} through \ref{sec:sellPhase} explain each game phase in detail.




\begin{figure}[p]
\begin{center}

\begin{graph}(7,8)(-3.5,-.5)
%\begin{framegraph}(6,5)(-3,0.5)
\graphnodesize{0.6}
\graphnodecolour{1}
\grapharrowlength{0.2}
\grapharrowwidth{0.75}
%\opaquetextfalse
%\fillednodesfalse

\newcommand{\phase}[4] { \textnode{#1}(#2,#3){{\large \begin{tabular}{c}#4\end{tabular}}} }    


\phase{pay}{0}{7}{Pay agents}

\phase{move}{0}{5.5}{Move agents\\(specify diamond bids)\\(specify inspector bribes)}

\phase{deposit}{0}{3.75}{Deposit diamonds\\(as agents move home)}


\phase{inspector}{0}{2.25}{Move UN inspector\\(inspector confiscates)}

\phase{collect}{0}{1}{Collect diamonds}

\phase{sell}{0}{0}{Sell diamonds}



\diredge{pay}{move}
\diredge{move}{deposit}
\diredge{deposit}{inspector}
\diredge{inspector}{collect}
\diredge{collect}{sell}


\dirbow{sell}{pay}{0.5}
\bowtext{sell}{pay}{0.5}{Repeat 9 times}


\end{graph}

\end{center}

%\caption{Phase progression in Diamond Trust}
\label{fig:gameFlow}

\end{figure}







\section{Controls}
The game is controlled entirely through single touches on the Touch Screen.  Various buttons and widgets appear on the Touch Screen when appropriate, and other situations require that you touch specific game elements directly.

\section{Hearing the music}

Diamond Trust's soundtrack features near-CD quality interactive music.  High quality headphones or external speakers are recommended.  

The music goes through transitions based on the decisions made by both players in the game.  Thus, it does not go through transitions when the game is idle, such as when the game is waiting on the title screen.  However, if you plug headphones or external speakers into the audio jack and then Close the System, the music will enter an auto-transition mode, allowing you to hear a fully-dynamic rendition of the soundtrack even while you're not actively playing the game.  Normal game-based music transitions will resume after you Open the System.

\section{Finding an opponent}

Diamond Trust is a game of skill for exactly two players.  You should eventually plan on playing the game exclusively against other people, but such match-ups are not ideal while you're still learning the game.  Maybe you could struggle along with another novice through your first game, but trying to learn while playing against and experienced player will likely be nerve-wracking for you and frustrating for your opponent.  The AI is provided to alleviate this problem.

\subsection{Practice against the AI}

If you touch ``Single Player'' from the title screen, you can pick an AI level and start a game against the AI.  The single player game works exactly the same way as a two player game, and the AI operates by the same rules as your human opponent would---it doesn't cheat by peaking at your secret information.

Unfortunately, reasoning about incomplete information requires quite a lot of ``possible worlds'' simulation on the part of the AI, which takes a long time in order to pick a good move.  At lower AI levels, the AI picks a move quickly but usually poorly.  At higher skill levels, the AI picks better moves but takes quite a bit longer.

Each additional level gives the AI more time to consider moves.  At level 2, it takes twice as long, while at level 7, it takes seven times as long.  But even at extremely high levels, where the AI is given an extraordinary amount of time to consider moves, human-level play is not possible.

Thus, the AI should be thought of as a practice opponent only.  Learn how to play the game, learn how to beat the AI at some of the lower skill levels, and then move on to a human opponent.  

\subsection{Single-Card Play}
\label{sec:singleCard}
To play against a friend, all you need is two DS systems and a single Diamond Trust Game Card.  By touching ``Two Player'' from the title screen, you can touch ``Host Single-Card Play'' serve the game to your friend's system via DS Download Play.  The only down-sides to this approach are the additional transmission and load time and the fact that the receiving player's game won't output music.

After the game has been served and loaded once, multiple games can be played against that friend.  It is even possible for two DS Download Play recipients to play subsequent games against each other.  Thus, a Diamond Trust party or tournament could be possible with only a few Game Cards shared by many people (see section \ref{sec:multiHosting} on page \pageref{sec:multiHosting} for more information).
 

\subsection{Multi-Card Play}
If both players have Diamond Trust Game Cards, they can connect to each other directly with no download time and both players experiencing the game's music.  After touching ``Two Player'' from the title screen, one player should touch ``Host Multi-Card Play'' while the other player touches ``Join Game.''

Headphones or two separate rooms are recommended to avoid unpleasant music combinations from the two systems.  Another option is to reduce the volume control to the minimum for one system and plug high-quality, external speakers into the audio jack for the second system, which will allow both players to listen to the music emanating from the second system.









\section{Game elements}

\subsection{Touch Screen elements:  Map and units}
Each player in the game is in control of three diamond trading agents.  The map is divided into eight regions:  six region in Angola, and one home region for each player (London and Antwerp, in the upper left and right corners of the screen, respectively).  Agents start the game in their home region, but they can move from region-to-region during the game.  At most one unit from each side is permitted in a given region of Angola.  Moving into the opponent's home region is forbidden.

The UN Inspector begins the game in a randomly chosen region of Angola.  The Inspector can be moved to other Angola regions throughout the game in response to player bribes, but he can never move into a player's home region.

Thus, a given home region can hold up to three agents from the same side.  A given region of Angola can hold up to three people:  one agent from each side and the UN Inspector.

The diamond icon in each Angola region indicates how many diamonds have accumulated there.

During the game, various buttons and widgets will appear on the Touch Screen when appropriate.  These include bid, bribe, and sale selectors, as well as buttons for finishing a move and advancing to the next game phase.

\subsubsection{In-game help}

During most phases of the game, a small ``?'' button in the lower right corner of the Touch Screen can be used to display relevant help text about the current phase.

\subsection{Upper screen elements:  Status displays}

Constant elements on the upper screen include the ledger sheets showing the money and diamond balance for each player (the opponent's money balance might be hidden, depending on what's going on in the game), the calendar showing the number of rounds remaining, and the green bar sheet at the bottom with information about the current game phase.

When an agent is selected during the game (when moving or paying that agent), a Rolodex card will appear showing known information about that agent.  Such information might include number of diamonds held, total salary, total bribes, the opponent agent known to have last bribed the agent, and other agents this agent is known to have bribed.  Of course, some of these elements may be unknown, depending on what's going on in the game (for example, an opponent agent's salary might be shown as ``?'' if that information has not been revealed to you).

During the Sell Diamonds phase of the game, additional ledger sheets are displayed showing the number sold and amount earned by each player.





\section{Game phases in detail}

\subsection{Game phase:  Pay agents}
\label{sec:payPhase}

During this phase, you can pay any of your own agents that are in your home region.  If your agents are out in the field, in Angola regions, they cannot be paid salary during this phase.  However, if one of your agent's out in the field occupies the same Angola region as an opponent agent, you can pay the opponent agent a bribe during this phase.

To determine which agents can be paid during this phase, look for agents that are waving their arms.  Touch an agent to adjust the salary or bribe payment to be made this round using the payment selection widget.  Touch the check mark on the widget to set the payment.  Before finishing your move, you can touch agents several times to adjust their salaries and bribes repeatedly.  Touch the ``Done'' button when to commit your payments. 

At the start of the game, all of your agents are at home, so you have the opportunity to pay salary to all three of them.  Since your agents aren't in contact with opponent agents, no bribes can be paid at the start of the game.

Salary that you pay to your agents keeps them loyal in the face of bribes they receive from your opponent.  Bribes paid to opponent agents might convert those agents' loyalty to your side.  If total salary payment meets or exceeds total bribes paid by the opponent to that agent, the agent remains loyal to its side.  However, if total bribes exceed total salary, the agent becomes loyal to the opponent.

Agents that have become loyal to the opponent reveal their hidden information secretly to the opponent.  Hidden information is described in greater detail in the following sections, but it includes move destinations, diamond bids, and inspector bribes, and number of diamonds selected to sell.

In particular, agents that have been successfully bribed reveal what other agents {\it they} have bribed, and for how much.  How much your own agents have been bribed is hidden from you, unless you successfully bribe the agent that bribed your agent---then you can find out.

\subsection{Game phase:  Move agents}

During this phase, you can specify destinations for each of your agents.  A trip inside Angola costs \$1 per agent.  A trip between Angola and Europe costs \$2 per agent.  The trip cost for the selected agent is shown at the top of that agent's Rolodex card.

To set an agent's destination, touch one of your agents and then touch an available destination region.  Unavailable regions are darkened after you pick an agent.  Remember that at most one of your agents can occupy each region of Angola.  You also need to be able to afford the trip cost.

If you pick an Angola region as a destination for an agent, you will then need to pick a bid for the diamonds in that region.  Use the bid selection widget to raise or lower your bid, and then touch the check mark to set the bid.

If you pick an Angola region occupied by the UN Inspector, you will get to specify a bribe payment for the Inspector after you specify your diamond bid.  The same widget is used to specify the Inspector's bribe.

All of your moves are specified at the same time as your opponent.  When you are satisfied with all of your chosen moves and payments, touch the ``Done'' button to commit your move.

\subsubsection{Moves revealed}
In the case where secret information is preserved for both sides, moves will be revealed and executed immediately after both players commit their moves.

However, in the case where some agents have been successfully compromised by opponent bribes, there will be a separate step where the moves of compromised agents are revealed and players get to adjust their moves before committing.  Of course, you only get to take advantage of this revealed information and chance to adjust if you have successfully bribed at least one of your opponent's agents.

This ``reveal and adjust'' step allows you to react to your opponent's chosen moves with better counter moves.  If you anticipate this step, it will also allow you to feed your opponent false information in your initial move selection.  Of course, you opponent might be doing the same thing to you.

During this step, players pick their final moves simultaneously and in secret.  Immediately after the reveal and adjust step, both final moves are revealed and executed.

\subsubsection{About diamond bids}
Auctions in Diamond Trust are somewhat unusual:  instead of a standard winner-pay auction mechanism, this game uses {\it both-pay auctions}.  If both players are sending an agent into a given region, they both specify bids, and both players pay the amount they have bid.  Then the highest bidder gets the diamonds in that region.  Even if players tie, both still pay their committed amounts, and neither player gets the diamonds.

\subsubsection{About inspector bribes}
Inspector bribes also follow the {\it both-pay} model.  The UN Inspector accepts bribe payments from both players, but only obeys the command of the player who pays him more.  Again, in the case of a tie, he takes money from both players but obeys neither player.

\subsection{Game phase:  Deposit diamonds}
After agent moves are executed, any agents that have returned to their home regions deposit any diamonds that they are carrying.  Diamonds deposited at home are permanently safe from confiscation by the UN inspector and are also available for sale during the Sell Diamonds phase.

\subsection{Game phase:  Move UN Inspector}

The UN Inspector has two functions in the game.  First, he blocks diamond acquisition in whatever region he occupies, effectively freezing the region and allowing diamonds to accumulate there.  Second, he confiscates diamonds carried by any agents that he encounters in his region.  However, if he will move this round due to a player's bribe, he only performs this confiscation after he moves, once he reaches his destination region.

After agents have moved and deposited any diamonds in home regions, the player who has paid a larger bribe to the UN Inspector gets to move the Inspector to a different region, if desired.  If neither player has paid him anything this round or both players have tied in terms of his payment, the Inspector does not move.

To move the Inspector, if you are the player who has paid him more, touch an Angola region as his destination.  You can switch regions by touching them.  When a region is selected, the upper screen shows information about what the Inspector will do in that region---the diamond sales to be blocked and the number of diamonds to be confiscated from each agent in that region.  When you are satisfied with the Inspector's selected destination, touch the ``Done'' button to make him move there.

After he moves, he performs his confiscation.  He also blocks whatever diamond sale was pending in his new region (though both players in that region still pay what they have bid for the diamonds there).  Thus, moving the Inspector can be used to open up an opportunity in one region while simultaneously shutting down an opportunity in another region.

\subsection{Game phase:  Collect diamonds}

After the Inspector's move, if he moves at all, and confiscation, agents in the other Angola regions collect diamonds according to the both-pay rule describe above.  Diamonds in regions where no player bids or both players tie simply remain in their region, accumulating for the next round.

Agents hold the diamond that they have collected, which are vulnerable to future Inspector confiscation, until they return to their home regions to deposit them.


\subsection{Game phase:  Sell diamonds}
\label{sec:sellPhase}

Before the start of the next round, players are given the chance to sell some of their accumulated diamonds---those that have been brought back to their home regions---one the open market to raise additional capital.

A player who sells no diamonds at all gets a flat \$18 injection of fresh money.  Players that chose to sell diamonds get a share of \$24 based on how many diamonds are sold in total, by both players, that round.  A player's share is determined by their fraction of the total sold.  If only one player sells, that player gets the entire \$24, no matter how many that player sells.  If both players sell the same amount, no matter how many they each sell, each player gets half of the available money, or \$12.  The following table illustrates this break-down with various examples:
\begin{center}
\begin{tabular}{r|r||r|r}
Player 1 sells & Player 2 sells & Player 1 gets & Player 2 gets\\
\hline
\hline
0 & 0 & \$18 & \$18 \\
\hline
1 & 0 & \$24 & \$18 \\
2 & 0 & \$24 & \$18 \\
3 & 0 & \$24 & \$18 \\
\hline
1 & 1 & \$12 & \$12 \\
2 & 1 & \$16 & \$8 \\
3 & 1 & \$18 & \$6 \\
\hline
1 & 2 & \$8 & \$16 \\
2 & 2 & \$12 & \$12 \\
3 & 2 & \$14 & \$9 \\
\end{tabular} 
\end{center}
Note that if the total sold does not divide 24 evenly, the amount paid to each player is rounded down (as shown in the last line of the above table).


\section{End of the game}

During the last round of the game, when the calendar display on the upper screen shows ``0 months left'', the round ends after agents collect diamonds in their regions of Angola.  There is no final Sell Diamonds phase.  All agents immediately fly back to their home regions and deposit whatever diamonds they are carrying.  These final trips home are free of travel costs.

The player with the most diamonds at the end of the game wins.  In the case of a tie in diamond counts, the player with the most money wins.  In the case of a tie in both diamond counts and money balances---a rare occurrence, for sure---a true tie is declared.



\section{Playing again}
The host of the game must decide whether to start another game after a game ends.  If the host touches the ``Back'' button, both players will be disconnected and returned to the title screen, where they can each play against the AI or host/join games with other game partners.

As mentioned in section \ref{sec:singleCard}, if the host chooses not to play again during DS Download Play, the cardless player will still be returned to the title screen where she will have full access to all game features, including hosting and joining.  The only thing a cardless player cannot do is host DS Download Play for another DS system that still needs to download the game.  A cardless player can continue playing the game indefinitely, hosting and joining subsequent games, until she turns the power OFF.


\section{Multi-hosting tricks}
\label{sec:multiHosting}
The hosting, joining, and DS Download Play serving system in Diamond Trust is extremely flexible and allows several interesting play setups.

A single Game Card can be used to distribute the game to multiple cardless DS systems without necessarily starting games with any of them.  To use this trick, host Single-Card Play normally, and after the first cardless system receives the game and displays the ``Loading'' screen with the advancing credit page, touch the ``Back'' button on the host system to cancel hosting and return to the title screen.  The connecting system will of course be unable to connect to the host after loading, so touch the ``Back'' button on that system as well.  Now you have two systems that are at the title screen and ready to host or join games.

Repeat this process for additional cardless systems, transferring the game to each through DS Download Play, but backing out before the cardless system finishes loading the game.

You will now have a room full of friends who all have downloaded the game and are ready to play games against each other, all from a single Game Card.

If multiple people want to join chosen partners in the same room, they should pair up and host/join one pair at a time.  The default behavior of Diamond Trust when multiple people are hosting and joining simultaneously is to pair them up randomly with each other (there is no lobby view for selecting among multiple available hosts).  This behavior can be used, if half of the friends host and the other half join simultaneously, to play against a friend whose identity is unknown to you, adding yet another layer of uncertainty and deception.



\part{Original design document}
\addtocounter{chapter}{1}
\setcounter{section}{0}


\documentclass[12pt]{article}
\usepackage{fullpage,times}

%\pagestyle{empty}

\newcommand{\gtitle}{{\it Diamonds}}

\newcommand{\rc}{resource}
\newcommand{\rcs}{resources}
\newcommand{\Rc}{Resource}
\newcommand{\Rcs}{Resources}


\title{Design concept: \gtitle\ (working title)}
\author{Jason Rohrer}


\begin{document}

\maketitle


\begin{abstract}
\gtitle\ for the Nintendo DSi is a two-player, turn-based strategy game about the dealings, secrets, and espionage of two diamond trading corporations.  The game takes place in Angola, a nation rich with diamonds but torn by factional conflict.  Each player controls a corporation that deploys hired agents into the region to purchase diamonds.  Agents maintain certain corporate secrets about their dealings, but their loyalty is for sale.  \gtitle\ is a game of incomplete information where opponents can secretly obtain information about each other.  
\end{abstract}


\section{Thematic overview}

The finest diamonds in the world are hidden here.  Not hidden, actually, but dug out of the earth by the teeming masses---bare backs and bare feet, knee deep in the rust-colored mud.  The diamonds are hidden only to those who are too afraid to come here, too afraid of conflict.  These diamonds are just waiting to be plucked, begging to be sold to whoever has the means of carrying them out to the rest of world.  Expensive diamonds out there, but cheap diamonds here.  What use are diamonds, anyway, in Angola?

The local market has been soft ever since the biggest player went back to Johannesburg.   Pressure from the media lead to pressure from the public.  The public buys diamonds, so the biggest player had to listen.  But we fly under the public's radar.  We're not 100-year-old dinosaurs.  We're made-up corporations with made-up names.  Low overhead.  No permanent employees, just field agents hired as needed.  Even our offices are temporary.  Temporary, but in the right locations.  Antwerp and London---the only places that matter.

There are only two firms left with agents in Angola.  We've gotten good takes while it lasted, but big changes are coming.  At first we thought the Kimberley Process was just a toothless promise from the big dealers to soothe public outcry.  Certify the origin of rough diamonds?  Sure, we can handle that.  But now we're hearing whispers about the UN. \ Seems like Kimberley may get herself some teeth in December.  But hey, that's December.  It's January now, and we've still got our contacts in Angola.  Let's make this last year count.


\section{Gameplay summary}

Two players each control a team of corporate agents that move between diamond-producing regions.  Unoccupied regions accumulate diamonds during each round, making them more attractive to occupy in future rounds.  Accumulated diamonds are sold at a flat rate to a single agent, but they are sold by auction if two opposing agents move into the region simultaneously.  Moves and bids are specified simultaneously in secret, so one strategic aspect involves deciding when to move into a region as more and more diamonds accumulate under the same flat price.  If both players jump into a region at the same time, a potentially costly auction may result.

A player's agents may be secretly bribed by the opponent.  Bribed agents reveal their secret information, such as move designations and bids, to the opponent.  Bribing costs money, though, so another strategic aspect involves deciding which of your opponent's agents to bribe and which of your own agents to trust.  Bribing the right opponent agents can reveal to you which of your own agents have been bribed, perhaps without your opponent knowing that you know.  You can then use the supposedly secret actions of these agents to mislead your opponent.  For example, if you know your opponent can see your secret bid, you might bid up an auction to push your opponent toward paying a higher price, then let your opponent win.

Collected diamonds can be converted into money through an open market sale process involving secret choices that are also subject to exposure through agent bribes.  Money earned through diamond sales can be used to purchase more diamonds, protect units from bribes, or bribe more opponent units.  The game lasts for a fixed number of rounds, and the player with the most money at the end of the final round wins.


\section{Mechanical overview}

Play takes place on a map of Angola divided into 8 diamond-producing regions.  Two additional regions, Antwerp and London, serve as the home bases for each player's corporation.  Each player has 5 units that represent their corporate agents.  All units start in their home base regions.  Players each start with 50 units of money.  Money balances can go negative or positive throughout the game with no limitation.  Each player's current money balance is secret information.

\subsection{Rounds}

The game takes place across 12 rounds, each representing a month in the year 2000 before the UN passes the Kimberley Process Certification Scheme in December.  During each round, players carry out various simultaneous actions in a particular order.  The order of actions are summarized as follows:
\begin{enumerate}

\item Diamonds accumulate in each region.
\item Players adjust salaries and bribes.
\item Players move their units.
\item Diamonds are sold or auctioned in occupied regions.
\item Players sell diamonds on the open market. 
\end{enumerate}
The following subsections provide details about each action.  


\subsubsection{Diamond accumulation}

At the start of each round, additional diamonds accumulate in each of the 8 regions.  Different regions accumulate diamonds at different rates, though the rates are set at the start of the game and never change.  The following distribution of accumulation rates applies:
\begin{itemize}
\item Three regions accumulate 1 diamond per round.
\item Two regions accumulate 2 diamonds per round.
\item Two regions accumulate 3 diamonds per round.
\item One region accumulates 4 diamonds per round.
\end{itemize}
Each region displays its accumulation rate, along with the total of uncollected diamonds that have accumulated there.


\subsubsection{Salaries and bribes}

After diamond accumulation, players can adjust the salaries of any units that are in their home base regions.  All units start there, so all salaries can be adjusted at the start of the first round.  To adjust the salary of a unit later in the game, the unit must return to the home base.  Salaries are specified in money units, and there is no limit on a given unit's salary.  The salary is paid from the player's balance every round.  Salaries for each unit in the game are always visible to both players.  The main purpose of salaries is to prevent a unit from being bribed, which is described below.

Any unit that occupies the same region as an opponent unit can secretly subject the opponent unit to a bribe.  After adjusting salaries for units in their own home base, a player can adjust bribes for any opponent units that they occupy regions with.  Bribes are set in money units, and like salaries, they are paid from the bribing player's balance every round.  Bribes are only effective if they are greater than the salary being paid by the unit's owner.  For example, a unit with a salary of 5 can only be successfully bribed with a payment of 6 or more.

Once a unit has been bribed, it shares all secret information with the other player.  The bribed status of the unit is hidden from the unit's owner.  The bribe may be negated by raising the unit's salary up to or above the bribe amount.

Ineffective bribes---those that are too low---are still paid each round.  They become effective instantly if the unit's salary is ever lowered below the bribe amount.

To adjust the bribe on a unit, an opponent unit must co-occupy a region with that unit again.  Any of the opponent's units may be used to adjust a bribe, not just the unit that originally set the bribe.  Whatever unit last adjusted the bribe becomes the bribing unit.

\subsubsection{Moving units}

After adjusting salaries and bribes, both players simultaneously select moves for their units.  Moving units during each round is optional, and each unit moved costs 1 money unit.  Units can jump from their current region to any region on the map in a single move.  Units can move back to their home region from producing regions, but they may never move into the opponent's home region.

At most one unit from each player can occupy a given producing region.  Units can occupy a producing region alone, or they can co-occupy the region with one opponent unit.  Any number of units can occupy a home region.  Moves are made using the following sequence:
\begin{enumerate}
\item Players specify their moves simultaneously, in secret.
\item Bribed units secretly reveal their destination to the opponent.
\item Players modify their moves simultaneously, in secret.
\item Moves are committed.
\end{enumerate}

\subsubsection{Diamond sale and auction}

\paragraph{Sale}
Along with its diamond accumulation rate, each region is also assigned a {\it flat rate price}.  This is the price for which all accumulated diamonds can be bought if there is only one lone unit present in the region.  If no unit is present in a region, the diamonds simply accumulate.  A unit that is alone in a region automatically buys all accumulated diamonds at the flat rate price.  Thus, the longer a region remains unoccupied, the more diamonds accumulate for the same total price, and the more attractive the region becomes.  Flat rate prices are fixed at random, at the start of the game, using the following distribution:
\begin{itemize}
\item Three regions sell their accumulated diamonds for 5 money units.
\item Two regions sell their accumulated diamonds for 4 money units.
\item Two regions sell their accumulated diamonds for 3 money units.
\item One region sells its accumulated diamonds for 2 money units.
\end{itemize}
Each region displays its flat rate price.

\paragraph{Auction}
If two opposing units are present in a region, the accumulated diamonds are put up for auction.  Bids are specified in money units with no upper limit.  An auction is conducted using the following sequence:
\begin{enumerate}
\item Players specify their bids simultaneously, in secret.
\item Bribed units secretly reveal their bids to the opponent.
\item Players modify their bids simultaneously, in secret.
\item The high bid wins.
\item The winner pays the lower bid, plus 1, for the accumulated diamonds.
\end{enumerate}
In the case of a tie, neither unit wins, and the diamonds simply accumulate into the next round.

Units carry whatever diamonds they purchase.  The number of diamonds carried by any unit is visible to both players.  When units are moved back to their home bases, they automatically deposit their diamonds there.  The total balance of diamonds at each home base is visible to both players.


\subsubsection{Converting diamonds to cash}
At the end of each round, players can sell diamonds from their home balance on the open market in exchange for money units.  There are 100 money units available each round, and these are split between the two players based on how many diamonds they each decide to sell.  If neither player sells any diamonds in the round, the money is forfeited.  If only one player sells diamonds, that player gets all 100 money units, no matter how many diamonds that player sells.  If both players sell, they get a fraction of the money units according to their fraction of the total diamonds sold.  If one player sells 5 diamonds and the other sells 8, then the first player gets 38 money units and the second player gets 61 (fractions are always rounded down to the nearest whole money unit).  

The number of units to sell is specified simultaneously in secret.  If a bribed unit is present in a player's home base, however, the player's chosen quantity is visible to the opponent.  Thus, selling diamonds on the open market is conducted using the following sequence: 
\begin{enumerate}
\item Players specify their quantities to sell, simultaneously, in secret.
\item Bribed units in the home bases secretly reveal the quantities to the opponent.
\item Players modify their quantities simultaneously, in secret.
\item Players are paid based on their fraction of the total sold.
\end{enumerate}

\subsection{Common play pattern}
All three of the gameplay systems that involve possibly-revealed hidden information use the same 4-step play pattern:
\begin{enumerate}
\item Select.
\item Peek.
\item Adjust.
\item Commit.
\end{enumerate}
{\it Peek} means obtaining information from bribed units, and {\it Adjust} means modifying your selection based on the new information obtained from peeking.  This play pattern will feel unfamiliar to new players, but its reuse throughout different phases of the game will make it easier to learn.

\subsection{Hidden information}
Units maintain various pieces of hidden information throughout the game.  This information is secretly revealed to the opponent, without the unit's owner knowing, if the unit is successfully bribed.  Hidden information includes:
\begin{enumerate}
\item Move destination.
\item Bids.
\item Which opponent units this unit has bribed.
\item Bribe amounts for each bribed opponent unit.
\end{enumerate}
By bribing an opponent's unit, a player can find out which of his or her units have been bribed, and for how much.  A player can use this information to counteract the bribes by raising salaries.

The home base region also maintains hidden information, including the current money balance and the number of diamonds selected for sale at the end of a round.  This information is revealed to the opponent if a bribed unit is present in the home base region.



\subsection{Win condition}
The player with the most money (or least negative balance) at the end of the 12th round is the winner.  Unsold diamond balances at the end of the 12th round are worthless.


\section{Platform}

\gtitle\ is designed for the Nintendo DSi.  The bottom screen shows the map and unit locations.  Stylus input is used to specify moves, salaries, bribes, bids, and sale quantities.  The top screen shows context-sensitive information about selected units and regions, as well as a full map overview.

\subsection{Multiplayer}

The game makes use of ad hoc WiFi for 2-player games between two nearby DSi units.  The game also supports single player against various skill levels of AI.

\subsection{Camera integration}
The game makes use of the DSi's cameras in the following ways:
\begin{enumerate}

\item During an auction, if your unit's bid is being secretly revealed to your opponent during auction step 2, your DSi's inner camera will be used to secretly snap a picture of your face.  This picture will be sent across the WiFi connection and displayed to your opponent.  The same applies for step 2 of the diamond-selling process.

\item If you have successfully bribed the opponent unit that bribed your unit, so that you know your unit has been bribed, you will be secretly informed when you are about to have your picture taken.  ``Smile, you're being watched!'' \  The picture will still be sent to your opponent, who may not know that you know you're being watched.  Thus, you need to avoid smiling in the photo and maintain a bluffing face.  The same applies during the diamond-selling process.

\end{enumerate}

\section{References}

\subsection{Conceptual references}
The mechanical concepts of \gtitle\ are primarily inspired by the work of Joe Halpern on reasoning about knowledge and uncertainty.  A particularly important touch-point is the Muddy Children Puzzle, a thought experiment with surprising implications.


\subsection{Thematic references}
\begin{itemize}
\item The UN's Fowler Report, which gives an account of the De Decker Brothers' activities in Angola.  Published in 2000.\\
http://www.un.org/News/dh/latest/angolareport\_eng.htm

\item A chronology of Angola history.\\
http://www.c-r.org/our-work/accord/angola/chronology.php
\end{itemize}

\end{document}


\part{Credits}
\addtocounter{chapter}{1}
\setcounter{section}{0}

\subsubsection{Jason Rohrer}
Concept, game design, programming, graphics, pixel fonts, package design, package photography, manual contents.

\subsubsection{Tom Bailey}
Interactive music direction, composition, arrangement, and production.

\subsubsection{Thanks}
Gun Consulting brought indiePub and Diamond Trust together.  Mark Seremet believed in this crazy project.  Lauren Serafin helped to playtest several early versions of the game.  The font used on the title screen is Oxford by Roger White.  The satellite image of Angola is from the NASA Visible Earth project.








\end{document}


