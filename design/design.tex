\documentclass[12pt]{article}
\usepackage{fullpage,times}

%\pagestyle{empty}

\newcommand{\gtitle}{{\it Diamonds}}

\newcommand{\rc}{resource}
\newcommand{\rcs}{resources}
\newcommand{\Rc}{Resource}
\newcommand{\Rcs}{Resources}


\title{Design concept: \gtitle\ (working title)}
\author{Jason Rohrer}


\begin{document}

\maketitle


\begin{abstract}
\gtitle\ for the Nintendo DSi is a two-player, turn-based strategy game about the dealings, secrets, and espionage of two diamond trading corporations.  The game takes place in Angola, a nation rich with diamonds but torn by factional conflict.  Each player controls a corporation that deploys hired agents into the region to purchase diamonds.  Agents maintain certain corporate secrets about their dealings, but their loyalty is for sale.  \gtitle\ is a game of incomplete information where opponents can secretly obtain information about each other.  
\end{abstract}


\section{Thematic overview}

The finest diamonds in the world are hidden here.  Not hidden, actually, but dug out of the earth by the teeming masses---bare backs and bare feet, knee deep in the rust-colored mud.  The diamonds are hidden only to those who are too afraid to come here, too afraid of conflict.  These diamonds are just waiting to be plucked, begging to be sold to whoever has the means of carrying them out to the rest of world.  Expensive diamonds out there, but cheap diamonds here.  What use are diamonds, anyway, in Angola?

The local market has been soft ever since the biggest player went back to Johannesburg.   Pressure from the media lead to pressure from the public.  The public buys diamonds, so the biggest player had to listen.  But we fly under the public's radar.  We're not 100-year-old dinosaurs.  We're made-up corporations with made-up names.  Low overhead.  No permanent employees, just field agents hired as needed.  Even our offices are temporary.  Temporary, but in the right locations.  Antwerp and London---the only places that matter.

There are only two firms left with agents in Angola.  We've gotten good takes while it lasted, but big changes are coming.  At first we thought the Kimberley Process was just a toothless promise from the big dealers to soothe public outcry.  Certify the origin of rough diamonds?  Sure, we can handle that.  But now we're hearing whispers about the UN. \ Seems like Kimberley may get herself some teeth in December.  But hey, that's December.  It's January now, and we've still got our contacts in Angola.  Let's make this last year count.


\section{Gameplay summary}

Two players each control a team of corporate agents that move between diamond-producing regions.  Unoccupied regions accumulate diamonds during each round, making them more attractive to occupy in future rounds.  Accumulated diamonds are sold at a flat rate to a single agent, but they are sold by auction if two opposing agents move into the region simultaneously.  Moves and bids are specified simultaneously in secret, so one strategic aspect involves deciding when to move into a region as more and more diamonds accumulate under the same flat price.  If both players jump into a region at the same time, a potentially costly auction may result.

A player's agents may be secretly bribed by the opponent.  Bribed agents reveal their secret information, such as move designations and bids, to the opponent.  Bribing costs money, though, so another strategic aspect involves deciding which of your opponent's agents to bribe and which of your own agents to trust.  Bribing the right opponent agents can reveal to you which of your own agents have been bribed, perhaps without your opponent knowing that you know.  You can then use the supposedly secret actions of these agents to mislead your opponent.  For example, if you know your opponent can see your secret bid, you might bid up an auction to push your opponent toward paying a higher price, then let your opponent win.

Collected diamonds can be converted into money through an open market sale process involving secret choices that are also subject to exposure through agent bribes.  Money earned through diamond sales can be used to purchase more diamonds, protect units from bribes, or bribe more opponent units.  The game lasts for a fixed number of rounds, and the player with the most money at the end of the final round wins.


\section{Mechanical overview}

Play takes place on a map of Angola divided into 8 diamond-producing regions.  Two additional regions, Antwerp and London, serve as the home bases for each player's corporation.  Each player has 5 units that represent their corporate agents.  All units start in their home base regions.  Players each start with 50 units of money.  Money balances can go negative or positive throughout the game with no limitation.  Each player's current money balance is secret information.

\subsection{Rounds}

The game takes place across 12 rounds, each representing a month in the year 2000 before the UN passes the Kimberley Process Certification Scheme in December.  During each round, players carry out various simultaneous actions in a particular order.  The order of actions are summarized as follows:
\begin{enumerate}

\item Diamonds accumulate in each region.
\item Players adjust salaries and bribes.
\item Players move their units.
\item Diamonds are sold or auctioned in occupied regions.
\item Players sell diamonds on the open market. 
\end{enumerate}
The following subsections provide details about each action.  


\subsubsection{Diamond accumulation}

At the start of each round, additional diamonds accumulate in each of the 8 regions.  Different regions accumulate diamonds at different rates, though the rates are set at the start of the game and never change.  The following distribution of accumulation rates applies:
\begin{itemize}
\item Three regions accumulate 1 diamond per round.
\item Two regions accumulate 2 diamonds per round.
\item Two regions accumulate 3 diamonds per round.
\item One region accumulates 4 diamonds per round.
\end{itemize}
Each region displays its accumulation rate, along with the total of uncollected diamonds that have accumulated there.


\subsubsection{Salaries and bribes}

After diamond accumulation, players can adjust the salaries of any units that are in their home base regions.  All units start there, so all salaries can be adjusted at the start of the first round.  To adjust the salary of a unit later in the game, the unit must return to the home base.  Salaries are specified in money units, and there is no limit on a given unit's salary.  The salary is paid from the player's balance every round.  Salaries for each unit in the game are always visible to both players.  The main purpose of salaries is to prevent a unit from being bribed, which is described below.

Any unit that occupies the same region as an opponent unit can secretly subject the opponent unit to a bribe.  After adjusting salaries for units in their own home base, a player can adjust bribes for any opponent units that they occupy regions with.  Bribes are set in money units, and like salaries, they are paid from the bribing player's balance every round.  Bribes are only effective if they are greater than the salary being paid by the unit's owner.  For example, a unit with a salary of 5 can only be successfully bribed with a payment of 6 or more.

Once a unit has been bribed, it shares all secret information with the other player.  The bribed status of the unit is hidden from the unit's owner.  The bribe may be negated by raising the unit's salary up to or above the bribe amount.

Ineffective bribes---those that are too low---are still paid each round.  They become effective instantly if the unit's salary is ever lowered below the bribe amount.

To adjust the bribe on a unit, an opponent unit must co-occupy a region with that unit again.  Any of the opponent's units may be used to adjust a bribe, not just the unit that originally set the bribe.  Whatever unit last adjusted the bribe becomes the bribing unit.

\subsubsection{Moving units}

After adjusting salaries and bribes, both players simultaneously select moves for their units.  Moving units during each round is optional, and each unit moved costs 1 money unit.  Units can jump from their current region to any region on the map in a single move.  Units can move back to their home region from producing regions, but they may never move into the opponent's home region.

At most one unit from each player can occupy a given producing region.  Units can occupy a producing region alone, or they can co-occupy the region with one opponent unit.  Any number of units can occupy a home region.  Moves are made using the following sequence:
\begin{enumerate}
\item Players specify their moves simultaneously, in secret.
\item Bribed units secretly reveal their destination to the opponent.
\item Players modify their moves simultaneously, in secret.
\item Moves are committed.
\end{enumerate}

\subsubsection{Diamond sale and auction}

\paragraph{Sale}
Along with its diamond accumulation rate, each region is also assigned a {\it flat rate price}.  This is the price for which all accumulated diamonds can be bought if there is only one lone unit present in the region.  If no unit is present in a region, the diamonds simply accumulate.  A unit that is alone in a region automatically buys all accumulated diamonds at the flat rate price.  Thus, the longer a region remains unoccupied, the more diamonds accumulate for the same total price, and the more attractive the region becomes.  Flat rate prices are fixed at random, at the start of the game, using the following distribution:
\begin{itemize}
\item Three regions sell their accumulated diamonds for 5 money units.
\item Two regions sell their accumulated diamonds for 4 money units.
\item Two regions sell their accumulated diamonds for 3 money units.
\item One region sells its accumulated diamonds for 2 money units.
\end{itemize}
Each region displays its flat rate price.

\paragraph{Auction}
If two opposing units are present in a region, the accumulated diamonds are put up for auction.  Bids are specified in money units with no upper limit.  An auction is conducted using the following sequence:
\begin{enumerate}
\item Players specify their bids simultaneously, in secret.
\item Bribed units secretly reveal their bids to the opponent.
\item Players modify their bids simultaneously, in secret.
\item The high bid wins.
\item The winner pays the lower bid, plus 1, for the accumulated diamonds.
\end{enumerate}
In the case of a tie, neither unit wins, and the diamonds simply accumulate into the next round.

Units carry whatever diamonds they purchase.  The number of diamonds carried by any unit is visible to both players.  When units are moved back to their home bases, they automatically deposit their diamonds there.  The total balance of diamonds at each home base is visible to both players.


\subsubsection{Converting diamonds to cash}
At the end of each round, players can sell diamonds from their home balance on the open market in exchange for money units.  There are 100 money units available each round, and these are split between the two players based on how many diamonds they each decide to sell.  If neither player sells any diamonds in the round, the money is forfeited.  If only one player sells diamonds, that player gets all 100 money units, no matter how many diamonds that player sells.  If both players sell, they get a fraction of the money units according to their fraction of the total diamonds sold.  If one player sells 5 diamonds and the other sells 8, then the first player gets 38 money units and the second player gets 61 (fractions are always rounded down to the nearest whole money unit).  

The number of units to sell is specified simultaneously in secret.  If a bribed unit is present in a player's home base, however, the player's chosen quantity is visible to the opponent.  Thus, selling diamonds on the open market is conducted using the following sequence: 
\begin{enumerate}
\item Players specify their quantities to sell, simultaneously, in secret.
\item Bribed units in the home bases secretly reveal the quantities to the opponent.
\item Players modify their quantities simultaneously, in secret.
\item Players are paid based on their fraction of the total sold.
\end{enumerate}

\subsection{Common play pattern}
All three of the gameplay systems that involve possibly-revealed hidden information use the same 4-step play pattern:
\begin{enumerate}
\item Select.
\item Peek.
\item Adjust.
\item Commit.
\end{enumerate}
{\it Peek} means obtaining information from bribed units, and {\it Adjust} means modifying your selection based on the new information obtained from peeking.  This play pattern will feel unfamiliar to new players, but its reuse throughout different phases of the game will make it easier to learn.

\subsection{Hidden information}
Units maintain various pieces of hidden information throughout the game.  This information is secretly revealed to the opponent, without the unit's owner knowing, if the unit is successfully bribed.  Hidden information includes:
\begin{enumerate}
\item Move destination.
\item Bids.
\item Which opponent units this unit has bribed.
\item Bribe amounts for each bribed opponent unit.
\end{enumerate}
By bribing an opponent's unit, a player can find out which of his or her units have been bribed, and for how much.  A player can use this information to counteract the bribes by raising salaries.

The home base region also maintains hidden information, including the current money balance and the number of diamonds selected for sale at the end of a round.  This information is revealed to the opponent if a bribed unit is present in the home base region.



\subsection{Win condition}
The player with the most money (or least negative balance) at the end of the 12th round is the winner.  Unsold diamond balances at the end of the 12th round are worthless.


\section{Platform}

\gtitle\ is designed for the Nintendo DSi.  The bottom screen shows the map and unit locations.  Stylus input is used to specify moves, salaries, bribes, bids, and sale quantities.  The top screen shows context-sensitive information about selected units and regions, as well as a full map overview.

\subsection{Multiplayer}

The game makes use of ad hoc WiFi for 2-player games between two nearby DSi units.  The game also supports single player against various skill levels of AI.

\subsection{Camera integration}
The game makes use of the DSi's cameras in the following ways:
\begin{enumerate}

\item During an auction, if your unit's bid is being secretly revealed to your opponent during auction step 2, your DSi's inner camera will be used to secretly snap a picture of your face.  This picture will be sent across the WiFi connection and displayed to your opponent.  The same applies for step 2 of the diamond-selling process.

\item If you have successfully bribed the opponent unit that bribed your unit, so that you know your unit has been bribed, you will be secretly informed when you are about to have your picture taken.  ``Smile, you're being watched!'' \  The picture will still be sent to your opponent, who may not know that you know you're being watched.  Thus, you need to avoid smiling in the photo and maintain a bluffing face.  The same applies during the diamond-selling process.

\end{enumerate}

\section{References}

\subsection{Conceptual references}
The mechanical concepts of \gtitle\ are primarily inspired by the work of Joe Halpern on reasoning about knowledge and uncertainty.  A particularly important touch-point is the Muddy Children Puzzle, a thought experiment with surprising implications.


\subsection{Thematic references}
\begin{itemize}
\item The UN's Fowler Report, which gives an account of the De Decker Brothers' activities in Angola.  Published in 2000.\\
http://www.un.org/News/dh/latest/angolareport\_eng.htm

\item A chronology of Angola history.\\
http://www.c-r.org/our-work/accord/angola/chronology.php
\end{itemize}

\end{document}